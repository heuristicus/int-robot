\documentclass[conference]{IEEEtran}
\usepackage{algorithm}
\usepackage{algpseudocode}

\title{Robot Assignment}
\author{951926 \and 1001231 \and 1024072 \and 1028907}
\date{\today}

\begin{document}
\maketitle

\begin{abstract}
  In this report, we present our solution to the 1996 AAAI Mobile Robotics Competition task ``Call a meeting''\cite{AAAIcomp}. We provide some background information about the areas of robotics that are relevant to the problem, with brief reference to literature. We then provide a detailed description of our system and evaluate its performance. Finally, we discuss the experimental results and suggest areas for improvement.
\end{abstract}
\section{Background}
\subsection{Robot Motion}
holonomic robots, PID control, reactive control
\subsection{Localisation}
\begin{itemize}
\item odometry issues
\item beliefs, prior and posterior $\overline{bel}$
\item sensor model, update
\item motion model
\end{itemize}
The aim of localisation is to obtain an estimate of the position of the robot at any given time. While it is possible to use odometry data to do so, this data is inherently noisy and prone to error. In particular, odometry errors can arise from changes in the surface being traversed and the robot's weight, among others. It is possible to mitigate the effect of these errors on estimates of the robot's position by using Bayesian techniques. 
\begin{algorithm}
  \caption{Basic Monte Carlo Localisation\cite{thrun}}
  \label{alg:basicMCL}
  \begin{algorithmic}[1]
    \State \textbf{Algorithm MCL}\textnormal{($\mathcal{X}_{t-1}, u_t, z_t, m$)}
    \State $\bar{\mathcal{X}}_t=\mathcal{X}_t=\emptyset$
    \For{$m=1$ to $M$}
    \State $x_t^{[m]}=\textbf{sample\_motion\_model}(u_t,x_{t-1}^{[m]})$
    \State $w_t^{[m]}=\textbf{sensor\_model}(z_t,x_t^{[m]},m)$
    \State $\bar{\mathcal{X}}_t=\bar{\mathcal{X}}_t+\langle x_t^{[m]},w_t^{[m]}\rangle$
    \EndFor
    \For{$m=1$ to $M$}
    \State \textnormal{draw $i$ with probability $\propto w_t^{[m]}$}
    \State \textnormal{add $x_t^{[i]}$ to $\bar{\mathcal{X}}_t$}
    \EndFor
    \State \textbf{return} $\bar{\mathcal{X}}_t$
  \end{algorithmic}
\end{algorithm}
Bayes filter, Kalman filter, particle filter (MCL), small section about mapping---still an active area of research in robotics. Mention SLAM, which has been pretty much solved.
\subsection{Route Planning}
\begin{algorithm}
  \caption{Probabilistic Road Map Generation}
  \label{alg:prm}
  \begin{algorithmic}[1]
    \State \textbf{Algorithm generate\_PRM}\textnormal{($map$)}
    \State $V = $\textbf{ sample\_vertices}\textnormal{($map$)}
    \For{$v_c\in V$}
    \While{\textnormal{$c(v_c)<C$}}
    \State $v_t =$\textbf{ get\_closest}\textnormal{($V$)}
    % phi is connectedInNeighbourhood, gamma is connectedInFreeSpace
    \If{$d(v_c,v_t)<D_n$}
    \If{$\neg \phi (v_c,v_t) \wedge \gamma (v_c,v_t)$}
    \State \textbf{connect}\textnormal{($v_c, v_t$)}
    \EndIf
    \Else
    \EndIf
    \EndWhile
    \EndFor
  \end{algorithmic}
\end{algorithm}
\begin{algorithm}
  \caption{Path Flattening}
  \label{alg:pathflat}
  \begin{algorithmic}[1]
    \State \textbf{Algorithm flatten\_path}\textnormal{($P, I, map$)}
    \For {$i := 0$ \textbf{to} $I$}
    \For {$j := 0$ \textbf{to} $\left|P\right|-2$}
    \State $A \gets $\textnormal{newpath($i$)}
    \State $B \gets $\textnormal{newpath($i+1$)}
    \State $C \gets $\textnormal{newpath($i+2$)}
    \If{freely\_connected(map,$A,C$)}
    \State $P\setminusB$
    \EndIf
    \EndFor
    \EndFor
    \State \textbf{return}\textnormal{ newpath}
  \end{algorithmic}
\end{algorithm}
PRM (sampling methods, graph search), RRT
\subsection{Exploration}
frontier based techniques
\subsection{Robot Vision}
\section{Design}
\subsection{System Structure}
MENTION ALGORITHM COMPLEXITY!
brief ROS description, callback based system, finite state automaton
\subsection{Platform}
Stuff about the pioneer---available sensors, some data about its size, specifications, our additions to it. Kinect specs. Include a picture of the robot with the kinect on it. 
\section{Experimentation}
\subsection{PRM}
inflated map - show inflated map superimposed onto the original map
Redo experiment for sampling methods. short, medium, long path length. Display image of map with one of the routes displayed and show the difference between the sampling methods. Find the optimum route by sampling a massive number of vertices on to the space and then finding a route using that---the flattened path is then the most optimal route, and we compare the other routes to this route for each experiment.

Redid sampling experiments - now have comparison for neighbourhood, threshold and nearestN strategies on 4 different paths, including trying to get into the corridor. 5,10,20,30 maxconn for each strategy, inflation radius of 5 so that you can just about get into the corridor. random vertices:25,50,100,200,400,800, grid step 0.5,1,2,4, cell size 1,2,4,6, target per cell 2,5

\textbf{REDO THIS ONCE THE ABOVE DATA IS COMPILED}Best sampling: (approximate) optimum path found for each path by using grid sampling with a step of 0.2m and 50 max connections and neighbourhood connection strategy. Screenshots available in screenshots dir. Repeated experiments five times for cell and random, once only for grid. 5,10,20,30 maxconnections. Used results from previous experiments on the sampling strategy, but compared the best parameters for each. Also checked whether the corridor which causes issues was accessible when using each strategy as a measure of its ability to populate tight spaces. Not necessarily a good evaluation, since grid may be good on this map by accident, but terrible on others. new_prmlogs contains data.


\subsection{Vision}
\subsection{Exploration}
Coverage experiments info: cell and grid done for cell size and grid spacing of 10,8,6,4,2,1, with cell repeated three times for each. The fov max distance was 3.5, angle 57, minimum distance 0.3. Started at 2.80,18.97,40 in stage. Max move speed 0.7m/s, max rotation speed 0.4 rad/sec. Ran until the end of the exploration path was reached.
\section{Discussion}
\subsection{Performance}
\subsection{Potential Improvements}
the path generated to do exploration could be improved by using heuristic techniques
\subsection{Conclusions}
\bibliographystyle{ieeetr}
\bibliography{report}
\end{document}
