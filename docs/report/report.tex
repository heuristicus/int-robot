\documentclass[conference]{IEEEtran}
\usepackage{algorithm}
\usepackage{algpseudocode}
\usepackage{amsmath}

\newcommand{\BigO}[1]{\ensuremath{\operatorname{O}\left(#1\right)}}

\title{WRITE SOMETHING CLEVER}
\author{951926 \and 1001231 \and 1024072 \and 1028907}
\date{\today}

\begin{document}
\maketitle

\begin{abstract}
  In this report, we present a system which performs the 1996 AAAI Mobile Robotics Competition task ``Call a meeting''\cite{AAAIcomp}. The robot was tasked with bringing a certain number of people to a meeting room previously determined as available. To complete the task, we use probabilistic road mapping, Monte Carlo localisation and a face detection module using the ROS framework, running on a Pioneer 2DX platform modified with a tripod-mounted Kinect. In the first section, we provide some background information about the areas of robotics that are relevant to the problem. We then provide a detailed description of our system and evaluate its performance. Finally, we discuss the experimental results and suggest areas for improvement.
\end{abstract}
\section{Background}
Solutions to complex tasks often require the use of multiple techniques to solve different sub-problems, and this task is no exception. We were required to implement a localisation algorithm to determine the position of the robot, a probabilistic road map to allow paths to be planned through the space, a navigation algorithm for path planning, a method of exploring the space, and some way of detecting people. We then had to implement a system which would combine all of these separate modules into a single system that would complete the task. In this section we present some background information about techniques that are often used to solve these problems.
\subsection{Localisation}
The aim of localisation is to obtain an estimate of the position of a mobile robot using sensor data \cite{localisation}. While it is possible to use odometry data to do so, this data is inherently noisy and prone to error. In particular, odometry errors can arise from changes in the surface being traversed and the robot's weight, among others. An important point to note is that localisation can be performed with a prepared map, or generating a map in real-time. The much more complex latter problem is called simultaneous localisation and mapping, which we will not discuss here. See \cite{slam} for an introduction to the SLAM problem.

\subsubsection{Bayes Filter}
Many advanced localisation techniques are based on the Bayes filter, which uses a belief distribution $bel(x_t)$ to represent the state $x_t$. The calculation of $bel(x_t)$ at time $t$ is dependent on $bel(x_{t-1})$ at time $t-1$, the last action $u_t$, and the last measurement $z_t$. In the first step, called the prediction step, the prior belief $\overline{bel}(x_t)$ is calculated. This step merges two probability distributions; the prior belief over the previous state $x_{t-1}$, and the probability of transitioning from that state to the posterior state $x_t$ given that the action $u_t$ was taken. This step does not take into account any measurement taken in the posterior state, predicting based solely on the knowledge of the action taken. In the second step, the measurement update, the posterior belief is calculated by multiplying the prior belief with the probability of being in the posterior state given that the measurement $z_t$ was observed. The result of this multiplication is generally not a probability, and therefore requires normalisation using some constant $\alpha$. As there are usually multiple posterior states, $x_t$ is usually a state vector rather than a single state, and so the two steps will be applied multiple times in order to update the belief for each state being considered. The filter is recursive, requiring some idea of the initial belief $bel(x_0)$ at time $t=0$. The initial belief is either a distribution centred on $x_0$, in the case where the initial position is known, or a uniform distribution over the space otherwise.
\begin{algorithm}
  \caption{Bayes filter \cite{thrun}}
  \label{alg:bayesfilter}
  \begin{algorithmic}[1]
        \State \textbf{Algorithm Bayes\_filter}\textnormal{($bel(x_{t-1}), u_t, z_t$)}
        \For{\textnormal{all} $x_t$}
        \State $\overline{bel}(x_t)=\int P(x_t\mid u_t, x_{t-1})bel(x_{t-1})dx_{t-1}$
        \State $bel(x_t)=\alpha P(x_t \mid z_t)\overline{bel}(x_t)$
        \EndFor
        \State \Return $bel(x_t)$
  \end{algorithmic}
\end{algorithm}
As the Bayes filter is not restricted to finite state spaces, it is not possible to implement it for anything other than very simple problems. There are two families of algorithms for localisation, known as \emph{recursive state estimators}, with various properties that permit the use of the Bayes filter in more complex estimation tasks \cite{thrun}.
\subsubsection{Gaussian Filters}
The basic principle of the family of Gaussian filters is the use of multivariate normal distributions, which can be formulated from a mean $\mu$ and covariance $\Sigma$, to represent belief. As a result, the assumption that the system is a linear Gaussian system is made; the initial belief must be a Gaussian, and both the state transition function and measurement probability must be linear functions. Although Gaussian filters can be extended to non-linear systems, they perform best when the system meets the assumptions made. One of the main advantages of such filters is the computational complexity, which is polynomial with respect to the dimensionality of the state space. The main disadvantage is that Gaussians are unimodal and therefore cannot represent situations in which there are multiple hypotheses; a situation that is often encountered in robotics. Examples include the Kalman filter and the information filter, which are derived from two different ways of representing Gaussians. Both of these filters can be extended to non-linear systems by using a Taylor expansion to produce linear approximations of non-linear functions. Mixtures of Gaussians can also be used to extend the Kalman filter to encompass situations in which multiple hypotheses are required, but each extension increases the complexity of the algorithm. This extensibility is one of the reasons for the popularity of the Kalman filter in state estimation problems. With some extension, the information filter is particularly suited to multi-robot systems where information from multiple sources must be integrated, but issues with complexity have resulted in the Kalman filter becoming the more popular of the two for the majority of problems \cite{thrun}.
\subsubsection{Nonparametric Filters}
In contrast to Gaussian filters, nonparametric filters discretise the probability distribution and do not require assumptions of linearity and Gaussian belief distributions. Instead, the state is approximated by a finite number of values which are taken from the belief state at any given time. The number of these values can be varied, and nonparametric filters converge to the correct state as the number tends towards infinity. Because this family of filters does not impose restrictions on the posterior density, they are useful for problems such as global localisation, which require the state to be represented in a complex form. Global localisation is the problem of determining the position of the robot without knowing its initial position, resulting in global uncertainty and the need for a multimodal belief distribution. Although as a result nonparametric filters are more computationally expensive than Gaussian filters, it is possible to vary the number of values used to represent the belief to suit the problem using \emph{adaptive} techniques. Examples of these types of filters are the histogram and particle filters. The histogram filter decomposes a continuous space into some finite number of regions, each of which is assigned a probability based on the belief. Extensions include using dynamic decomposition techniques to use more coarse representations in regions with lower probability, and the use of \emph{selective updating}, which updates only the parts of the space which are deemed important. Particle filters approximate the belief by drawing a number of hypotheses called \emph{particles} randomly from the belief distribution. The most important part of the particle filter is the resampling step, which selects particles from the initial set proportional to an importance factor. This has the effect of concentrating particles in areas of high likelihood, reducing computational power spent in areas which are not relevant. Improving the sampling method and adapting the number of particles based on time and uncertainty lead to more effective and error-resistant particle filters. A property which is particularly useful to us is that particle filters are very easy to implement \cite{thrun}.

\begin{algorithm}
  \caption{Basic Monte Carlo Localisation \cite{thrun}}
  \label{alg:basicMCL}
  \begin{algorithmic}[1]
    \State \textbf{Algorithm MCL}\textnormal{($\mathcal{X}_{t-1}, u_t, z_t, map$)}
    \State $\bar{\mathcal{X}}_t=\mathcal{X}_t=\emptyset$
    \For{$m=1$ to $M$}
    \State $x_t^{[m]}=\textbf{sample\_motion\_model}(u_t,x_{t-1}^{[m]})$
    \State $w_t^{[m]}=\textbf{sensor\_model}(z_t,x_t^{[m]},map)$
    \State $\bar{\mathcal{X}}_t=\bar{\mathcal{X}}_t+\langle x_t^{[m]},w_t^{[m]}\rangle$
    \EndFor
    \For{$m=1$ to $M$}
    \State \textnormal{draw $i$ with probability $\propto w_t^{[m]}$}
    \State \textnormal{add $x_t^{[i]}$ to $\bar{\mathcal{X}}_t$}
    \EndFor
    \State \textbf{return} $\bar{\mathcal{X}}_t$
  \end{algorithmic}
\end{algorithm}

\subsubsection{The Markov Assumption}
One of the reasons that these filters are so efficient stems from the assumption that the Markov property holds. While this may not actually be the case, the Bayes filter that forms the base of these filters is robust to the violation of some of the requirements of the property. The Markov property states that the future states of the system do not depend on past states. In other words, considering all previous states of the system provides no additional information for predicting future states; all prediction can be done using only the current state. Previous states do not have to be included in calculations, and as a result do not need to be stored, which leads to faster computation and lower storage space requirements.

\subsubsection{Sensor \& Motion Models}
To actually use any of these filters to solve a localisation problem, it is necessary to model the sensors and motion of the robot. These models include some parameters representing the uncertainty attached to performing a specific action or receiving a certain measurement from a sensor. The motion model is used in the prediction step to determine the state transition probability $P(x_t\mid x_{t-1}, u_t)$; the probability of going from state $x_{t-1}$ to $x_t$ given that the action $u_t$ was performed. The sensor model is used to calculate $P(z_t\mid x_t,map)$, which represents the likelihood of the measurement $z_t$ being received in the state $x_t$ on a given map. The model incorporates knowledge about the noise parameters of the operating environment and the sensor being used in the form of probability densities. For example, a range sensor might include densities for correct measurements, measurements of unexpected obstacles, sensor failures and random measurements \cite{thrun}. 

Integrating the sensor and motion models into the particle filter results in an algorithm known as Monte Carlo localisation, otherwise known as MCL, the most basic version of which is shown in Algorithm \ref{alg:basicMCL}, in which $\mathcal{X}_{t-1}$ represents the particles from the previous time step, $M$ the total number of particles, $x_t^{[m]}$ the state of the $m$th particle with the action $u_t$ applied to it via the motion model, $w_t^{[m]}$ the importance weight of the $m$th particle, and $\bar{\mathcal{X}}_t$ the resampled set of particles. We use an adaptive implementation of MCL provided by the ROS system.

\subsection{Route Planning}
\subsubsection{Probabilistic Road Mapping}\cite{prm}
The probabilistic roadmap planner is a motion planning algorithm in robotics, which solves the problem of determining a path between a starting configuration of the robot and a goal configuration while avoiding collisions.
The general idea is to place random points into the configuration space of the robot and check them whether they are in a free space and not colliding with objects etc. Then a planner is used to connect these points to other nearby neighbours (within distance). Once they have been connected, the start and goal configurations are added in and a path planner is applied to determine a path between start and goal node. There are a variety of path planners available to use, such as Dijkstra's, A* search etc.

There are two phases to PRM: Construction phase and Query Phase. In the construction phase, a graph is built. While building the graph it is ensured that nodes are connected legally, i.e. illegal lines (lines through objects) are ignored. There are multiple ways in which neighbours can be connected, i.e. via the use of k nearest neighbours or neighbours a certain distance from each other. The next stage is the query phase in which the start and end nodes are added to the graph and some form of search algorithm is used to find a path between the two nodes.

The textbook Robotics Modelling, Planning and control give a good explanation of PRM. We will explain in summary what is explained in this book below.

In the beginning of PRM, you generate a random sample qrand of the configuration space using a uniform probability distribution in C. Then qrand is tested for collisions with objects within the configuration space. If no qrand is placed in legal places, they are then joined to one another, if and only if the edges joining them do not cause a collision with an object on the map. This is done via the use of an algorithm. Edges are joined by checking for near configurations where near is defined by a Euclidean distance in C.  The generation of a free local path between qrand and a near configuration qnear is delegated to a procedure known as local planner. A common choice according to the textbook is to throw a rectilinear path in C between qrand and qnear and test it for collision, for example by sampling the segments and checking them for collision. If the local path causes a collision, it is disregarded and therefore there is no direct path between the two points.
The PRM incremental generation procedure stops when a maximum number of iterations have been reached. When this point is reached, one checks to see if it is possible to solve the initial problem of finding a path between qs and qg via the use of the connected free paths.
If a solution is not found, the PRM can be improved by performing more iterations of the incremental generation procedure.

The main advantage of PRM according to the textbook is the speed of PRM. It can find solutions to motion planning problems in a very quick time. Another advantage is its simplicity to implement.
A disadvantage of PRM is that it is probabilistically complete, i.e. the probability of finding a solution to the problem when one exists tends to 1 as the execution time tends to infinity. To avoid reaching infinity, it is usual practice to incorporate a maximum number of iterations.
\subsubsection{Rapidly Exploring Random Trees}\cite{rrt}

\begin{algorithm}
  \caption{Probabilistic Road Map Generation}
  \label{alg:prm}
  \begin{algorithmic}[1]
    \State \textbf{Algorithm generate\_PRM}\textnormal{($map$)}
    \State $V = $\textbf{ sample\_vertices}\textnormal{($map$)}
    \For{$v_c\in V$}
    \While{\textbf{connections}\textnormal{$(v_c)<C$}}
    \State $v_t =$\textbf{ get\_closest}\textnormal{($V$)}
    % phi is connectedInNeighbourhood, gamma is connectedInFreeSpace
    \If{$d(v_c,v_t)<D_n$}
    \If{$\neg \phi (v_c,v_t) \wedge \gamma (v_c,v_t)$}
    \State \textbf{connect}\textnormal{($v_c, v_t$)}
    \EndIf
    \Else
    \EndIf
    \EndWhile
    \EndFor
  \end{algorithmic}
\end{algorithm}
\begin{algorithm}
  \caption{Path Flattening}
  \label{alg:pathflat}
  \begin{algorithmic}[1]
    \State \textbf{Algorithm flatten\_path}\textnormal{($P, I, map$)}
    \For {$i := 0$ \textbf{to} $I$}
    \For {$j := 0$ \textbf{to} $\left|P\right|-2$}
    \State $A \gets $\textnormal{newpath($i$)}
    \State $B \gets $\textnormal{newpath($i+1$)}
    \State $C \gets $\textnormal{newpath($i+2$)}
    \If{freely\_connected(map,$A,C$)}
    \State \textbf{remove}($P,B$)
    \EndIf
    \EndFor
    \EndFor
    \State \textbf{return}\textnormal{ newpath}
  \end{algorithmic}
\end{algorithm}
PRM (sampling methods, graph search), RRT
\subsection{Exploration}
frontier based techniques
\subsection{Robot Vision}
\section{Design}
\subsection{System Structure}
MENTION ALGORITHM COMPLEXITY!
brief ROS description, callback based system, finite state automaton
\subsection{Platform}
Stuff about the pioneer---available sensors, some data about its size, specifications, our additions to it. Kinect specs. Include a picture of the robot with the kinect on it. 
\section{Experimentation}
\subsection{PRM}
inflated map - show inflated map superimposed onto the original map
Redo experiment for sampling methods. short, medium, long path length. Display image of map with one of the routes displayed and show the difference between the sampling methods. Find the optimum route by sampling a massive number of vertices on to the space and then finding a route using that---the flattened path is then the most optimal route, and we compare the other routes to this route for each experiment.

Redid sampling experiments - now have comparison for neighbourhood, threshold and nearestN strategies on 4 different paths, including trying to get into the corridor. 5,10,20,30 maxconn for each strategy, inflation radius of 5 so that you can just about get into the corridor. random vertices:25,50,100,200,400,800, grid step 0.5,1,2,4, cell size 1,2,4,6, target per cell 2,5

\textbf{REDO THIS ONCE THE ABOVE DATA IS COMPILED}Best sampling: (approximate) optimum path found for each path by using grid sampling with a step of 0.2m and 50 max connections and neighbourhood connection strategy. Screenshots available in screenshots dir. Repeated experiments five times for cell and random, once only for grid. 5,10,20,30 maxconnections. Used results from previous experiments on the sampling strategy, but compared the best parameters for each. Also checked whether the corridor which causes issues was accessible when using each strategy as a measure of its ability to populate tight spaces. Not necessarily a good evaluation, since grid may be good on this map by accident, but terrible on others. new\_prmlogs contains data.


\subsection{Vision}
\subsection{Exploration}
Coverage experiments info: cell and grid done for cell size and grid spacing of 10,8,6,4,2,1, with cell repeated three times for each. The fov max distance was 3.5, angle 57, minimum distance 0.3. Started at 2.80,18.97,40 in stage. Max move speed 0.7m/s, max rotation speed 0.4 rad/sec. Ran until the end of the exploration path was reached.
\section{Discussion}
\subsection{Performance}
\subsection{Potential Improvements}
the path generated to do exploration could be improved by using heuristic techniques
\subsection{Conclusions}
\bibliographystyle{ieeetr}
\bibliography{report}
\end{document}
