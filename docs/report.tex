\documentclass[conference]{IEEEtran}
\usepackage{algorithm}
\usepackage{algpseudocode}

\title{Robot Assignment}
\author{951926 \and 1001231 \and 1024072 \and 1028907}
\date{\today}

\begin{document}
\maketitle

\begin{abstract}
  In this report, we present our solution to the 1996 AAAI Mobile Robotics Competition task ``Call a meeting''\cite{AAAIcomp}. We provide some background information about the areas of robotics that are relevant to the problem, with brief reference to literature. We then provide a detailed description of our system and evaluate its performance. Finally, we discuss the experimental results and suggest areas for improvement.
\end{abstract}
\section{Background}
\subsection{Robot Motion}
holonomic robots, PID control, reactive control
\subsection{Localisation}
\begin{itemize}
\item odometry issues
\item beliefs, prior and posterior $\overline{bel}$
\item sensor model, update
\item motion model
\end{itemize}
The aim of localisation is to obtain an estimate of the position of the robot at any given time. While it is possible to use odometry data to do so, this data is inherently noisy and prone to error. In particular, odometry errors can arise from changes in the surface being traversed and the robot's weight, among others. It is possible to mitigate the effect of these errors on estimates of the robot's position by using Bayesian techniques. 
\begin{algorithm}
  \caption{Basic Monte Carlo Localisation\cite{thrun}}
  \label{alg:basicMCL}
  \begin{algorithmic}[1]
    \State \textbf{Algorithm MCL}\textnormal{($\mathcal{X}_{t-1}, u_t, z_t, m$)}
    \State $\bar{\mathcal{X}}_t=\mathcal{X}_t=\emptyset$
    \For{$m=1$ to $M$}
    \State $x_t^{[m]}=\textbf{sample\_motion\_model}(u_t,x_{t-1}^{[m]})$
    \State $w_t^{[m]}=\textbf{sensor\_model}(z_t,x_t^{[m]},m)$
    \State $\bar{\mathcal{X}}_t=\bar{\mathcal{X}}_t+\langle x_t^{[m]},w_t^{[m]}\rangle$
    \EndFor
    \For{$m=1$ to $M$}
    \State \textnormal{draw $i$ with probability $\propto w_t^{[m]}$}
    \State \textnormal{add $x_t^{[i]}$ to $\bar{\mathcal{X}}_t$}
    \EndFor
    \State \textbf{return} $\bar{\mathcal{X}}_t$
  \end{algorithmic}
\end{algorithm}
Bayes filter, Kalman filter, particle filter (MCL), small section about mapping---still an active area of research in robotics. Mention SLAM, which has been pretty much solved.
\subsection{Route Planning}
\begin{algorithm}
  \caption{Probabilistic Road Map Generation}
  \label{alg:prm}
  \begin{algorithmic}[1]
    \State \textbf{Algorithm generate\_PRM}\textnormal{(map,)}
    \State $\mathbf{V} = $\textbf{ sample\_vertices}\textnormal{(map)}
    \While{\textnormal{connectionCount $<$ maxConnections}}
    \State $v =$\textnormal{get\_closest($\mathbf{V}$)}
    \If{\textnormal{$\neg$ connectedInNeighbourhood($v$)}}
    \If{\textnormal{connectedInFreeSpace($v, vert$)}}
    \State \textnormal{connect($v, vert$)}
    \EndIf
    \Else
    \If{\textnormal{$\neg$ connected($v, vert$)}}
    \If{\textnormal{connectedInFreeSpace($v, vert$)}}
    \State \textnormal{connect($v,vert$)}
    \EndIf
    \EndIf
    \EndIf
    \EndWhile
  \end{algorithmic}
\end{algorithm}
\begin{algorithm}
  \caption{Path Flattening}
  \label{alg:pathflat}
  \begin{algorithmic}[1]
    \State \textbf{Algorithm flatten\_path}\textnormal{(path, iterations, map)}
    \State \textnormal{newpath $\gets$ path}
    \For {$i := 0$ \textbf{to} iterations}
    \For {$j := 0$ \textbf{to} len(path)$-2$}
    \State $A \gets $\textnormal{newpath($i$)}
    \State $B \gets $\textnormal{newpath($i+1$)}
    \State $C \gets $\textnormal{newpath($i+2$)}
    \If{freely\_connected(map,$A,C$)}
    \State \textnormal{newpath.remove($B$)}
    \EndIf
    \EndFor
    \EndFor
    \State \textbf{return}\textnormal{ newpath}
  \end{algorithmic}
\end{algorithm}
PRM (sampling methods, graph search), RRT
\subsection{Exploration}
frontier based techniques
\subsection{Robot Vision}
\section{Design}
\subsection{System Structure}
MENTION ALGORITHM COMPLEXITY!
brief ROS description, callback based system, finite state automaton
\subsection{Platform}
Stuff about the pioneer---available sensors, some data about its size, specifications, our additions to it. Kinect specs. Include a picture of the robot with the kinect on it. 
\section{Experimentation}
\subsection{PRM}
inflated map - show inflated map superimposed onto the original map
Redo experiment for sampling methods. short, medium, long path length. Display image of map with one of the routes displayed and show the difference between the sampling methods. Find the optimum route by sampling a massive number of vertices on to the space and then finding a route using that---the flattened path is then the most optimal route, and we compare the other routes to this route for each experiment.
\subsection{Vision}
\subsection{Exploration}
\section{Discussion}
\subsection{Performance}
\subsection{Potential Improvements}
\subsection{Conclusions}
\bibliographystyle{ieeetr}
\bibliography{report}
\end{document}
